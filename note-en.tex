%%
%% In this note, we explain how to generate the low data in our paper
%% ``Holonomic gradient method for the probability content of a simplex region
%% under a multivariate normal distribution''
%%
%% To compile this TeX file, simply type the following command:
%%     latex note-en.tex && latex note-en.tex
%%

%% label 1,2

\documentclass[12pt]{article}

\textwidth 155 true mm
\textheight 247 truemm
\topmargin -0.4 true mm
\headheight 0 true mm
\headsep 0 true mm
\oddsidemargin 2.1 true mm
\evensidemargin 2.1 true mm
\footskip 5 true mm
%\titlesep 12pt

%\pagestyle{empty}

\usepackage{color}
%\usepackage{refcheck}
\usepackage{url}
%\usepackage{amssymb}
%\usepackage{amsmath}
%\usepackage{graphicx}
\usepackage{listings}

%\usepackage{amsthm}
%\newtheorem{theorem}{Theorem}
%\newtheorem{proposition}{Proposition}
%\newtheorem{lemma}{Lemma}
%\newtheorem{corollary}{Corollary}
%\newtheorem{conjecture}{Conjecture}
%\newtheorem{algorithm}{Algorithm}
%\theoremstyle{definition}
%\newtheorem{definition}{Definition}
%\newtheorem{example}{Example}

%% for shell script
\lstdefinestyle{BashInputStyle}{
  language=bash,
  firstline=2,% Supress the first line that begins with `%`
  basicstyle=\small\sffamily,
  numbers=left,
  numberstyle=\tiny,
  numbersep=3pt,
  frame=tb,
  columns=fullflexible,
  backgroundcolor=\color{white},
  linewidth=0.9\linewidth,
  xleftmargin=0.1\linewidth
}

\def\pd#1{\partial_{#1}}

\title{A Note for Numerical Experiments}
\author{Tamio Koyama}
\date{}

\begin{document}
\maketitle
%\thispagestyle{empty}

\begin{abstract}
In this note, we explain how to generate the low data in our paper
\cite{2015koyama2}.
\end{abstract}

\section{Preparation}
For our numerical experiments, we prepared 
C program {\tt program.c}, 
shell scripts \\{\tt numerical\_experiment.sh}, 
and {\tt R} scripts {\tt monte\_carlo.R}, {\tt dist\_func.R}.
Our computer system is linux(debian), and we utilized GNU scientific library
BLAS, and {\tt R}.
The source file {\tt program.c} can be compiled by the following command:
\begin{lstlisting}[style=BashInputStyle]

gcc program.c -lm -lgsl -lblas
\end{lstlisting}

 
In our numerical experiments, we generated the following files:
{\tt tab3.tex}, 
{\tt tab2.tex}, 
{\tt tab1.tex}, 
{\tt raw\_data4.txt}, 
{\tt raw\_data3.txt}, 
{\tt raw\_data2.txt}, 
{\tt raw\_data1.txt},
{\tt fig1.eps}.

\section{Synopsis}
\begin{verbatim}
./a.out S d [a_11 a_12 ... b_1 a_21 a_22 ... b_2 ...]
\end{verbatim}
Our program computes 
the probability content of a simplex region
with a multivariate normal distribution
$$
\int_{\mathbf R^d} \frac{1}{(2\pi)^{d/2}}
\exp\left(-\frac{1}{2}\sum_{i=1}^dx_i^2\right)
\prod_{j=1}^{d+1}H\left(f_j(a,b,x)\right)
dx
$$
and the probability content of a simplicial cone
$$
\int_{\mathbf R^d} \frac{1}{(2\pi)^{d/2}}
\exp\left(-\frac{1}{2}\sum_{i=1}^dx_i^2\right)
\prod_{j=1}^dH\left(f_j(a,b,x)\right)
dx.
$$
Here, $H(x)$ is the Heaviside function and we set 
$f_j(a,b,x)=\sum_{i=1}^da_{ij}x_i+b_j$.
\begin{itemize}
\item \verb+S+\/: Switch the behavior of the program. Behaviors are
  \begin{itemize}
    \item \verb+S=1+\ : %%    simplex();
    Compute the probability content of a simplex region by HGM
    \item \verb+S=2+\ : %%    simplicial_cone();
    Compute the probability content of a simplicial cone
    \item \verb+S=11+\ : %%    numerical_experiment1(p);
    Generate data for Table 1.
    \item \verb+S=12+\ : %%    numerical_experiment2(p);
    Generate data for Table 3.
    \item \verb+S=13+\ : %%    numerical_experiment3(p);
    Generate data for Table 2.
    \item \verb+S=14+\ : %%    numerical_experiment4(p);
    Generate data for Figure 1.
  \end{itemize}
\item \verb+d+\/: Dimension.
\item \verb+a_11,...+\/: Parameters $a_{ij}$.
\item \verb+b_1,...+\/: Parameters $b_i$.
\end{itemize}


\section{How to Generate Low Data}
The following command generates the low data
{\tt tab3.tex}, 
{\tt tab2.tex}, 
{\tt tab1.tex}, 
{\tt raw\_data4.txt}, 
{\tt raw\_data3.txt}, 
{\tt raw\_data2.txt},
{\tt raw\_data1.txt}, and
{\tt fig1.eps}:

\begin{lstlisting}[style=BashInputStyle]

gcc program.c -lm -lgsl -lblas
./numerical_experiment.sh
\end{lstlisting}


%\iffalse
\iftrue

\begin{thebibliography}{1}

\bibitem{2015koyama2}
T.~Koyama.
\newblock Holonomic gradient method for the probability content of a simplex
  region with a multivariate normal distribution.
\newblock \url{http://arxiv.org/abs/1512.06564}, 2015.

\end{thebibliography}

\else

\bibliographystyle{plain}
\bibliography{reference}

\fi

%\appendix
\end{document}
