%%
%% Note for program.c
%%

%% label 1,2
\documentclass[12pt]{article}

\textwidth 155 true mm
\textheight 247 truemm
\topmargin -0.4 true mm
\headheight 0 true mm
\headsep 0 true mm
\oddsidemargin 2.1 true mm
\evensidemargin 2.1 true mm
\footskip 5 true mm
%\titlesep 12pt

%\pagestyle{empty}

%\usepackage{color}
%\usepackage{refcheck}
%\usepackage{url}
%\usepackage{amssymb}
%\usepackage{graphicx}
\usepackage{framed}
\usepackage{amsmath}

\usepackage{amsthm}
%\newtheorem{theorem}{Theorem}
%\newtheorem{proposition}{Proposition}
%\newtheorem{lemma}{Lemma}
%\newtheorem{corollary}{Corollary}
%\newtheorem{conjecture}{Conjecture}
%\newtheorem{algorithm}{Algorithm}
\theoremstyle{definition}
%\newtheorem{definition}{Definition}
\newtheorem{example}{Example}

\def\pd#1{\partial_{#1}}

\title{Note for {\tt program.c}}
\author{Tamio Koyama}
\date{}

\begin{document}
\maketitle
\section{Preparations}
My code {\tt program.c} can be compiled by the following command:
\begin{framed}
\begin{verbatim}
$gcc program.c -lm -lgsl -lblas -llapack
\end{verbatim}
\end{framed}
Compiling {\tt program.c} needs {\tt GSL}, {\tt BLAS}, and {\tt LAPACK}.
The following results of numerical experiments were obtained 
on Linux(Debian).

\section{Simplex (the case of Moving parameter $b$ only)}
In this section, we show a numerical experiment in the case where
we do not move parameter $a$.
In other words, for given parameter $a$ and $b$, we take 
$$
a_0 = a,\,b_0 = 0
$$
as the initial point.

\if0
\begin{example}
Let the dimension be $d=2$.
Suppose the simplex is defined by 
\begin{align*}
  x_1 &\geq 0, & x_2&\geq 0, & x_1+x_2&\leq 10.
\end{align*}
Then the parameters are
\begin{align*}
  a &= \begin{pmatrix} 1 & 0 \\ 0 & 1 \\ -1 & -1  \end{pmatrix}^\top, &
  b &= \begin{pmatrix} 0 \\ 0 \\ 10 \end{pmatrix}.
\end{align*}
In this case, the probability can be obtained by the command
\begin{framed}
\begin{verbatim}
$./a.out 1 2  1 0 0  0 1 0  -1 -1 10
\end{verbatim}
\end{framed}
And we have the following result:
\begin{framed}
\begin{verbatim}
...
result:
   0.2499999984    0.4999999979    0.4999999979    1.0000000000   ...
Probability = 0.25
\end{verbatim}
\end{framed}
\end{example}
\fi

\begin{example}\label{2} %% Example from Xi san
Let the dimension be $d=3$.
Suppose the simplex is defined by
\begin{align*}
  x_1 &\geq 0,\\ 
  x_2 &\geq 0,\\
  x_3 &\geq 0,\\
  11x_1+12x_2+13x_3&\leq 10.
\end{align*}
Then the parameters are
\begin{align}\label{1}
  a &= 
  \begin{pmatrix} 
    1 & 0 & 0 \\ 
    0 & 1 & 0 \\ 
    0 & 0 & 1 \\ 
  -11 & -12 & -13
  \end{pmatrix}^\top, &
  b &= 
  \begin{pmatrix} 0 \\ 0 \\ 0 \\ 10 \end{pmatrix}.
\end{align}
In this case, the command
\begin{framed}
\begin{verbatim}
$./a.out 1  3  1 0 0 0  0 1 0 0  0 0 1 0 -11 -12 -13 10
\end{verbatim}
\end{framed}
computes the probability of the simplex, and we obtain the following result
\begin{framed}
\begin{verbatim}
...
result:
   0.0055573545    0.0459324799    0.0495911154    0.2791218357 ...
Probability = 0.00555735
\end{verbatim}
\end{framed}
\end{example}

\section{Simplex(the case of moving parameter $a$ and $b$)}
In this section, we show a numerical experiment in the case where
we move both parameter $a$ and $b$.
In this section, we fix the initial point as
$$
a_0 = 
\begin{pmatrix} 
  1 & 0 & \cdots & 0\\
  0 & 1 &        & 0\\
  \vdots & & \ddots & \\
  0 & 0 &        & 1\\
  -1 & -1 & \cdots & -1
\end{pmatrix}^\top
,\,b_0 = 0.
$$
\begin{example}
Let $d=3$ and suppose that parameters $a$ and $b$ are given by \eqref{1}.
Then the command 
\begin{framed}
\begin{verbatim}
$./a.out 2  3  1 0 0 0  0 1 0 0  0 0 1 0 -11 -12 -13 10
\end{verbatim}
\end{framed}
computes the probability of the simplex, and we have the following result:
\begin{framed}
\begin{verbatim}
...
result:
   0.0055573553    0.0459324836    0.0495911191    0.2791218390 ...
Probability = 0.00555736
\end{verbatim}
\end{framed}
We obtained the same value which we got in Example \ref{2}.
\end{example}

\end{document}
